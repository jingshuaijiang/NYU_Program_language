\documentclass[12pt]{article}

\usepackage[utf8]{inputenc}
\usepackage{latexsym,amsfonts,amssymb,amsthm,amsmath}
\usepackage{listings}

\lstset{
 basicstyle=\fontsize{7}\selectfont\ttfamily
 columns=fixed,                           % 设定行号格式
 frame=none,                                          % 不显示背景边框
 keywordstyle=\color[RGB]{40,40,255},                 % 设定关键字颜色
 numberstyle=\footnotesize\color{darkgray},           
 commentstyle=\it\color[RGB]{0,96,96},                % 设置代码注释的格式
 stringstyle=\rmfamily\slshape\color[RGB]{128,0,0},   % 设置字符串格式
 showstringspaces=false,                              % 不显示字符串中的空格
}

\setlength{\parindent}{0in}
\setlength{\oddsidemargin}{0in}
\setlength{\textwidth}{6.5in}
\setlength{\textheight}{8.8in}
\setlength{\topmargin}{0in}
\setlength{\headheight}{18pt}



\title{Program Language homework2}
\author{jingshuai jiang jj2903}

\begin{document}

\maketitle

\vspace{0.5in}



\subsection*{Exercise 1}

a). $[a-z]^*[A-Z][a-z]^*[0-9][a-z]^*[A-Z][a-z]^*$
\\[10pt]b).$([0-9]|[1-9][0-9]^*).[0-9][0-9]^*E([0-9]|[1-9][0-9]^*)$
\\[10pt]c).$[A-Z](\epsilon|[A-Za-z0-9\_]|[A-Za-z0-9\_][A-Za-z0-9\_]|[A-Za-z0-9\_][A-Za-z0-9\_][A-Za-z0-9\_]|[A-Za-z0-9\_][A-Za-z0-9\_][A-Za-z0-9\_][A-Za-z0-9\_]|[A-Za-z0-9\_][A-Za-z0-9\_][A-Za-z0-9\_][A-Za-z0-9\_][A-Za-z0-9\_]|[A-Za-z0-9\_][A-Za-z0-9\_][A-Za-z0-9\_][A-Za-z0-9\_][A-Za-z0-9\_][A-Za-z0-9\_])$




\subsection*{Exercise 2}
\begin{lstlisting}
    Prog => Parts
		|Parts Parts

    Parts => Dec
		|Fun
		|Proc

    Fun => fun name(P) Dec {content}

    Proc => proc name(P) Dec {content}

    content => content content
		   |E
		   |return name

    E=> EOE
	    |name(V)

    O => =
	    |-
	    |*
	    |/
	    |+

    V => name
	    |num
	    |V,V

    P => ϵ
	    |P,P
	    |Dec

    Dec => name:name
    
\end{lstlisting}


\clearpage
\subsection*{Exercise 3}

a). $static\ scoping:$ names used in a function are resolved in the environment of the function definition.
\\[10pt] $dynamic\ scoping:$ names used in a function are resolved in the environment of the function called.
\\[10pt]b).
\begin{lstlisting}
    begin
    integer m, n;

    procedure A;
        begin
        print("in A : n = ", n);
        end;

    procedure B(n: integer);
        begin
        print("in B : m = ", m);
        print("in B : n = ", n);
        hardy;
        end;

    m := 50;
    n := 100;
    print("in main program : n = ", n);
    B(1);
    A;
    end;
    
\end{lstlisting}
The static scoping outputs:
\\[10pt] in main program : n = 100
\\[10pt]in B : m = 50
\\[10pt]in B : n = 1
\\[10pt]in A : n = 100    
\\[10pt]in A : n = 100 

The dynamic scoping outputs:
\\[10pt] in main program : n = 100
\\[10pt]in B : m = 50
\\[10pt]in B : n = 1
\\[10pt]in A : n = 1    
\\[10pt]in A : n = 100

\vspace{1in}
\textbf{c).} It will resolve the variable reference fisrt in the function itself. Then it will look for the variable from the function where it was defined.

\vspace{1in}
\textbf{d).} It will resolve the variable reference fisrt in the function itself. Then it will look for the variable from the function where it was called.

\subsection*{Exercise 4}
\clearpage
\subsection*{Exercise 5}
a).pass by value: 2 4 6 8 10
\\[10pt]b).pass by value: 2 11 6 8 10
\\[10pt]c).pass by value: 2 7 6 8 10
\\[10pt]d).pass by value: 2 4 6 8 11

\subsection*{Exercise 6}
a).
\begin{lstlisting}
    
with text_io, ada.integer_text_io;
use text_io, ada.integer_text_io;

procedure mainprog is
package int_io is new integer_io (integer);

    use int_io;

    task oddtask is
        entry okdone;
    end oddtask;

    task eventask is
        entry okdone;
    end eventask;

    task body oddtask is
    begin
                accept okdone;
                for k in 1..100 loop
                ada.integer_text_io.put(k);
                if k rem 10 = 0 and k /=100 then
                eventask.okdone;
                accept okdone;
                end if;
                if k =100 then
                eventask.okdone;
                end if;
            end loop;
    end oddtask;

    task body eventask is
    begin
               accept okdone;
            for k in 201 .. 300 loop
              ada.integer_text_io.put(k);
              if k rem 10 = 0 and k /=300 then
              oddtask.okdone;
              accept okdone;
              end if;
            end loop;
    end eventask;

begin
        oddtask.okdone;
end mainprog;

\end{lstlisting}
\vspace{2in} %Leave more space for comments!







\end{document}
